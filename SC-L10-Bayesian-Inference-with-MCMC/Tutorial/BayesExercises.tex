\documentclass{article}
\addtolength{\oddsidemargin}{-.875in}
\addtolength{\evensidemargin}{-.875in}
\addtolength{\textwidth}{1.75in}

\addtolength{\topmargin}{-1in}
\addtolength{\textheight}{2in}
\usepackage{bm}
\begin{document}
\title{Statistical Computing: Bayesian Inference}
\date{June 12}
\maketitle
\section*{Markov chain Monte Carlo}
We will simulate a sample $\mu^{[1]},\mu^{[2]}\mu^{[3]},\ldots,\mu^{[M]}\sim p(\mu|{\bm y})$.  The kernel of the target density is
\begin{equation}
p(\mu|{\bm y})\propto \left[\sum\limits_{i=1}^n (y_i-\mu)^2\right]^{-n/2}
exp\left[-\frac{(\mu-\eta)^2}{2\tau^2}\right]
\end{equation}
where 
\begin{itemize}
\item $M=25000$
\item $n=10$
\item $\sum y_i=17.34$
\item $\sum y^2_i=32$
\item $\eta=1.8$
\item $\tau^2=0.25$
\end{itemize}
Use the Metropolis algorithm to simulate from this.  The steps are
\begin{enumerate}
\item Initialise $\mu^{[0]}=1.8$
\item Initialise an $M\times$1 vector for storing values of $\mu^{[j]}$
\item Find $\mu^{[1]}$ using the following steps
\begin{itemize}
\item Set $\mu^{old}$=$\mu^{[0]}$
\item Generate $\mu^{new}\sim N(\mu^{old},\nu^2)$ where $\nu^2$ is the variance of the proposal.  For now, set $\nu^2=0.02$.
\item Compute the acceptance probability 
\begin{equation}
\alpha=min\left(1,\frac{p(\mu^{new}|{\bm y})}{p(\mu^{old}|{\bm y})}\right)
\end{equation}
\item Generate an 0/1 indicator called {\em accept} where $\mbox{Pr}(accept=1)=\alpha$
\item If $accept=1$ then set $\mu^{[1]}=\mu^{[new]}$, otherwise $\mu^{[1]}=\mu^{[old]}$
\end{itemize} 
\item Use a loop for {\bf j} {\em in} 2:M.  Inside the  loop:
\begin{itemize}
\item Do the same as in step 3, but instead of updating $\mu^{[0]}\rightarrow \mu^{[1]}$, update $\mu^{[j-1]}\rightarrow \mu^{[j]}$
\end{itemize}
\end{enumerate}
\section*{Bayes Inference}
Using your sample from the first part, find an approximation of: 
\begin{enumerate}
\item The Posterior Mean $E[\mu|{\bm y}]$
\item The Posterior Median of $\mu|{\bm y}$
\item The Posterior Mode of $\mu|{\bm y}$ ({\em difficult})
\item A 95\% credible interval for $\mu|{\bm y}$
\item The posterior probability $\mbox{Pr}(\mu>1.7|{\bm y})$
\end{enumerate}
Don't forget to remove the first few iterates in the sample (burn-in).
\section*{Better Code}
Make the following changes to your code
\begin{itemize}
\item Write code to give a trace plot of your iterates. 
\item Write code that allows you to compute the acceptance rate (i.e. the percentage of times a proposed value of $\mu^{new}$ is accepted).
\end{itemize}
\section*{Assess scheme}
Instead of $\nu^2=0.02$, try half ($\nu^2=0.01$) this value. 
\begin{enumerate}
\item Do you expect the acceptance rate to change? Does it? Why? 
\item Do you expect the posterior mean to change? Does it? Why? 
\end{enumerate}
Do the same for ($\nu^2=0.04$).  
\section*{Preparation for Tuesday's Lecture}
Download the package CODA.  Try to use the functions {\em geweke.diag} and {\em EffectiveSize} on your Monte Carlo sample.
\end{document}