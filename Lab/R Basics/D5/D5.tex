\documentclass[a4paper]{article}
\usepackage{amsmath}
\usepackage{hyperref}
\usepackage{breakurl}
\usepackage{lineno}
\usepackage{graphicx}
\usepackage[a4paper]{geometry} 
\geometry{verbose,tmargin=2.5cm,bmargin=2.5cm,lmargin=2.5cm,rmargin=2.5cm}

\usepackage{fancyhdr}
%The first page setting
\fancypagestyle{plain}
{
  \fancyhf{} % clear current header and footer fields
  \fancyhead[L]
  {
    LINK\"OPING UNIVERSITY\\
    Division of Statistics\\
    Dept. of Computer and Information Science\\
    Feng Li \& Mattias Villani
  }
  \fancyhead[R]{Programming in R}
}
   
%The remaining pages
\pagestyle{fancy}
\fancyhead[RO,LE]{} 
\fancyhead[C]{Programming in R}
\fancyhead[LO,RE]{}
  
\title{D5--R packages, matrix computations, strings and time series}  
\date{March 19, 2012}

\begin{document}
\maketitle
\hrule
\begin{center}
\textbf{Instructions}
\end{center}
\begin{itemize}
\item You should always try hard to solve the problems yourself. You may
  discuss them with others but any sort of plagiarism is strictly forbidden.

\item Questions marked with ``\textbf{Extra}'' do not need to be handed in. 

\item Questions marked with $\heartsuit$ should be answered in your report. All
  other questions need not be in your submitted report. Make extensive use of
  comments (lines starting with the \texttt{\#}-symbol) in your code.

\item Submit your report to \emph{CourseKit} no later than \textbf{Mar 23,
    6:00pm}.
\end{itemize}
\hrule
  
\section{$\heartsuit$}
\begin{enumerate}
\item List all installed R packages on your system.

\item List all the packages that are loaded in your current R session.

\item Find a proper R package in the CRAN tasks view site
  (\url{http://cran.r-project.org/web/views/}) that includes the probability
  and quantile functions for the multivariate student distribution (also called the
  multivariate t distribution or the multivariate student t distribution), and install it on
  your system. [Hint: there are many to do this, just pick one of them]

\item Load the package that you have installed in \textbf{1.3}.

\item Use the \texttt{help()} function to find general information on the
  package that you have installed. Find the function in the loaded package that calculates the quantiles of the multivariate t distribution.
 
\item Use the \texttt{example()} function to run the examples for that
  function.

\item Remove the package you have loaded in \textbf{1.4} from R's working memory.

\item Uninstall the package from the system.

\item Recall the \texttt{mad()} function (the one that calculates the mean
  absolute deviation) that you defined in \textbf{Computer lab 3}. Run that function so
  that you have it in R's memory (check with ls() to see if it is indeed there after you
  run it). There is another function with the same name in the package \texttt{stats} that calculates the median
  absolute deviation. The \texttt{stats} package is always loaded into R's memory. How
  would you call (use) the function \texttt{mad()} in
  \texttt{stats} package without renaming or deleting your own function?

\end{enumerate}

\section{$\heartsuit$}
The dataset \texttt{AppleWithDate.csv} is identical to the Apple dataset we
used in \textbf{Computer lab 3 and 4} except it has an extra column
\texttt{Date} with the date of the observations. That is, an entry \texttt{2012-02-24} in the \texttt{Date} column
means Feb 24, 2004.

\begin{enumerate}
\item Load the dataset \texttt{AppleWithDate.csv} and name it \texttt{AppleNew}, and perform the following tasks:
  
  \begin{enumerate}
  \item Extract the columns \texttt{Open}, \texttt{High}, and
    \texttt{Low} and into a new matrix with the name \texttt{X0}.

  \item Create a $n\times 4$ ($n$ rows and $4$ columns) matrix where the entries in the first column are
    all ones ($1$) and the remaining columns contain the columns from \texttt{X0}. $n$ is
    number of observations. Name your matrix \texttt{X}.

  \item Extract the column \texttt{Close} as a $n\times 1$ matrix and name it \texttt{y}.

  \item Extract the column \texttt{Date} and convert it to a vector in \textbf{R's date format}. Save
    your it to a $n\times 1$ matrix and name it \texttt{Date}.

  \end{enumerate}


\item The linear model 
  \begin{equation*}
    \mathtt{Close} = \hat \alpha_0 + \hat \alpha_1 \cdot \mathtt{Open} + \hat
    \alpha_2 \cdot \mathtt{High}+
    \hat \alpha_3\cdot \mathtt{Low}
  \end{equation*}
  can also be estimated direct matrix calculations (as an alternative to lm()). Follow the following
  instructions to implement it. 

  \begin{enumerate}
  \item Transpose your matrix \texttt{X} and name it \texttt{XT}.

  \item Calculate the matrix product $X'X$ and name it as \texttt{XTX} where $X'$ is the transpose
    of $X$.

  \item Check if $X'X$ is invertible. [Hint: You can check if all
    eigenvalues of $X'X$ are greater than zero]

  \item Calculate the matrix inverse of $X'X$ and name the result as \texttt{XTXInv}

  \item Calculate the coefficients for the linear model via $\hat \alpha = (X'X)^{-1} X'y $ and
    save the result as a $4\times 1$ matrix and name it \texttt{alphaHat}.

  \item Run the linear model with \texttt{lm()} and compare the estimated coefficients to
    those in question 2(e) above.

  \item Calculate the fitted value $\hat y$ for $y$, i.e, $\hat y = X \hat
    \alpha$ and name it \texttt{yHat}.
 
  \item Calculate the estimated residuales $\hat \epsilon = y - \hat y$ and
    name it \texttt{eHat}.

  \end{enumerate}

\item Make a figure with $2\times 1$ subplots for the most recent 50 days where
  the first subplot includes a scatter plot of \texttt{Close} price against
  \texttt{Date} and a line plot of \texttt{yHat} price against \texttt{Date}
  with color in red, the second plot is \texttt{eHat} price against
  \texttt{Date} with blue line. Add corresponding labels to the axes.

\end{enumerate}


\section{$\heartsuit$}
Suppose you have a string

\begin{verbatim}
       myStr <- "Programming-in-R"
\end{verbatim}

\begin{enumerate}
\item Count the number of characters in the string.

\item Manipulate the string so that all the uppercase letters in the string are
  converted to lowercase. [Hint: Google it!]

\item Extract the first seven characters in the string.
  
\item Replace the dashes (-) in the string with spaces. [Hint: see the function gsub()]

\item Suppose you also have the following strings
\begin{verbatim}
Name <- "Kate"
Grade <- "C"
Year <- 2012
\end{verbatim}
Use the \texttt{paste()} function to construct the following sentence:
\begin{verbatim}
Kate got a C in the course Programming in R in 2012.
\end{verbatim}

\end{enumerate}

\section{}
\begin{enumerate}
\item  Install and load the \texttt{zoo} package on your system.

\item Convert the data in \texttt{AppleNew} into zoo class and save the object
  as \texttt{AppleTS}.

\item Plot the data object \texttt{AppleTS}. Set the line width as \texttt{c(1,
    1.5, 2, 2.5)}, \\ line type as
  \texttt{c("solid","dashed","dotted","longdash")},\\ and color as
  \texttt{c("red","blue","purple","grey")}.
 
\item Add a legend to the plot and save it to disk as pdf format.
\end{enumerate}
 
\section{}
 Let $m=4$ and $n=6$ and answer the following questions

\begin{enumerate}
  
  \item Generate a random vector from the uniform distribution of length
    $m*n$ and name it as \texttt{myVec}.

  \item Convert the vector into a $m \times n$ matrix and fill the entries of
    the matrix \textbf{by row} and name it as \texttt{X}.

  \item Generate a diagonal matrix where the diagonal elements are from the
    sequence \texttt{d <- seq(1,n)} and name it as \texttt{D}.

  \item Calculate the matrix multiplication \texttt{X\%*\%D} and save it as
    \texttt{outDirect}.

  \item Create a $m\times n$ matrix named \texttt{D2} where each row in the
    matrix is filled with the vector \texttt{d}.

  \item Calculate the element-wise multiplication \texttt{X*D2} and
    save it as \texttt{outInDirect}.

  \item Compare if the two matrices \texttt{outDirec}t and \texttt{outInDirect}
    are identical.

   \item Now let $m=400$ and $n=6000$ and redo \textbf{5.1-5.6}. Record the execution
  (running) time for calculating \textbf{5.4} and \textbf{5.6}.

\end{enumerate}

%\section{}
%Recall the data in the \texttt{AppleNew} in \texttt{2.1} and answer the
%following questions

% \begin{enumerate}
% \item Create a new matrix \texttt{Z} where the columns are \texttt{Open},
%   \texttt{High}, \texttt{Low}, \texttt{Close} in the dataset.

% \item Calculate the covariance matrix of \texttt{Z} and name it as
%   \texttt{covZ}. cov[Hint: \texttt{?cov}]

% \item Find the variances of \texttt{Open}, \texttt{High}, \texttt{Low},
%   \texttt{Close} from the covariance matrix \texttt{covZ} and convert it as
%   $4\times 1$ matrix with the name of \texttt{V}.
 
% % \item Create two new $4\times 4$ matrics \texttt{V} and \texttt{U} where all
% %   the rows of $V$ are filled by \texttt{varZ} and $U = V'$.


% \item Calculate the correlation matrix only using the covariance matrix you
%   have created in \textbf{6.2} as follows

%   \begin{enumerate}
%   \item Calculate $VV'$ and name the result as \texttt{U}.
    
%   \item Calculate the element-wise squared root of the matrix \texttt{U}. 

%   \item Divide \texttt{covZ} by \texttt{U} element-wise and save it as \texttt{corZ}.

%   \item Run \texttt{cor(Z)} to check your previous calculates are all right.
%   \end{enumerate}

% \item Extract the lower triangular part (without diagonal part) of the matrix
%   \texttt{corZ}.

% \item Check if \texttt{corZ} is symmetric. [Hint: symmetric matrix is a square matrix that is equal to its transpose]

% \item \textbf{Extra}: If you are only given the lower triangular part of the
%   correlation matrix how can your recover to the whole correlation matrix?

% \end{enumerate}



\end{document}
