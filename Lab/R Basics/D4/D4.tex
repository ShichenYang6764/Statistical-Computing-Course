\documentclass[a4paper]{article}
\usepackage{amsmath}
\usepackage{hyperref}
\usepackage{breakurl}
\usepackage{lineno}
\usepackage{graphicx}
\usepackage[a4paper]{geometry} 
\geometry{verbose,tmargin=2.5cm,bmargin=2.5cm,lmargin=2.5cm,rmargin=2.5cm}

\usepackage{fancyhdr}
%The first page setting
\fancypagestyle{plain}
{
  \fancyhf{} % clear current header and footer fields
  \fancyhead[L]
  {
    LINK\"OPING UNIVERSITY\\
    Division of Statistics\\
    Dept. of Computer and Information Science\\
    Feng Li \& Mattias Villani
  }
  \fancyhead[R]{Programming in R}
}
  
%The remaining pages
\pagestyle{fancy}
\fancyhead[RO,LE]{} 
\fancyhead[C]{Programming in R}
\fancyhead[LO,RE]{}
  
\title{D4 -- Statistics and Graphics}  
\date{March 8, 2012}

\begin{document}
\maketitle
\hrule
\begin{center}
\textbf{Instructions}
\end{center}
\begin{itemize}
\item You should always try hard to solve the problems yourself. You may
  discuss them with others but any sort of plagiarism is strictly forbidden.

\item Questions marked with ``\textbf{Extra}'' do not need to be handed in. 

\item Questions marked with $\heartsuit$ should be answered in your report. All
  other questions need not be in your submitted report. Make extensive use of
  comments (lines starting with the \texttt{\#}-symbol) in your code.

\item Submit your report to \emph{CourseKit} no later than \textbf{Mar 15,
    6:00pm}.
\end{itemize}
\hrule
  
\section{}
Generate the following types of random numbers.
\begin{enumerate}
\item Generate a random vector of length $30$ from the uniform distribution between $1$
  and $10$.

\item Generate a random vector of length $200$ from the normal distribution with
  mean $1$, variance $2$ and name it as \texttt{myRNormVec}.

\item Plot a histogram and a boxplot of \texttt{myRNormVec}.

\item Generate a random vector of length $30$ from the Poisson distribution with
  mean $3$ and name it as \texttt{myRPoisVec}.

\item Generate a random vector of length $30$ from the student-\emph{t}
  distribution with mean $1$, degrees of freedom of $5$.

\item Sample $5$ items without replacement from the sequence
  \texttt{1:10} (a result here could for example be the sample 2, 4, 1, 5 and 7). 

\item Let the vector \texttt{friends} contain the names of your three closest
  friends and also include your own name in the vector. You and your friends have two
  tasks that you need to accomplish (maybe organizing a joint dinner and cleaning up the kitchen afterwards). Randomly select who will do each of the two tasks. The same person may end
  up doing both tasks.

\item \textbf{Extra:} Redo \textbf{1.2}  two times. Do you get the same vector?
  Now, redo \textbf{1.2} another two times, but issue the command
  \texttt{set.seed(1234)} before generating the random numbers. Do you get the same vector?
  [Hint: see \texttt{?set.seed} for the details.]

\end{enumerate}

\section{$\heartsuit$}
 Assume that a random variable $X$ follows the normal distribution with mean $\mu=1$ and
 variance $\sigma^2=2$. Answer the following questions regarding the random variable: 

  \begin{enumerate}
  \item Calculate the density function (tathetsfunktion in Swedish) $p(x)$ in the point $x = 2$.

  \item Calculate the probability that $X \leq 2$.

  \item Plot the distribution function $P(X\leq x)$ over the domain $-10 \leq x \leq
    10$. [Hint use \texttt{seq()} to set up the domain]
 
  \item Use the command \texttt{dev.copy2pdf(file="NormCDFCurve.pdf")} to save the graph
    from the previous exercise in PDF format.

  \item Find the $95\%$ quantile of the distribution.

  \end{enumerate}


\section{$\heartsuit$}
\begin{enumerate}
\item Load the Apple dataset from \textbf{Computer lab 3} and name
  it \texttt{Apple}. Store the variable (column)
  \texttt{High} from \texttt{Apple} in a new vector named \texttt{appleHigh}. Assume that
  \texttt{appleHigh} is a random sample from a normal population with mean $\mu$ and
  variance $\sigma^2$. Assume that both $\mu$ and $\sigma^2$ are unknown.

\item Estimate $\mu$ and $\sigma^2$.

\item Calculate a two-sided 95\% confidence interval for $\mu$.  [Hint: ?\texttt{t.test}]


\end{enumerate}

\section{$\heartsuit$}
Load also the Google dataset from \textbf{Computer lab 3}.
\begin{enumerate}
\item Select the \texttt{Close} columns from the two datasets and name them
  \texttt{appleClose} and \texttt{googleClose} respectively. Create a new
  variable
\begin{verbatim}
time <- length(appleClose):1
\end{verbatim}
which indicates the time for the stock prices for the two companies (note that the most
recent stock prices are ordered first in the data vectors).

\item Make a scatter plot of \texttt{appleClose} against \texttt{time}
. Use ``time'' as the x-axis label, ``Prices'' as the
y-axis label and ``Stock prices'' as the main title. 

\item Redo \textbf{4.2} but now with a line plot instead of a scatter. Change the color of
  the line to \textbf{blue} and set the \texttt{ylim} argument
  to \texttt{ylim = c(0, max(appleClose, googleClose))}.

\item Add a new line \texttt{googleClose} to the plot in \textbf{4.3} and set
  the line color to \textbf{red} [Hint: use \texttt{points()} or \texttt{lines()}]

\item Add an informative legend to the plot in \textbf{4.4}.

\item Save your plot to a file in png format. [Hint: \texttt{?savePlot}]

\item Redo 4.1-4.5, but plot the Apple and Google data in a 1-by-2 subplot. That is, plot the
  two datasets side-by-side in separate axes. [hint:
  \texttt{par(mfrow=c(1,2))} sets up the plots in a 1-by-2 format]
 
\end{enumerate}

\section{}
\begin{enumerate}
\item It is believed that the close price for the stock market may be affected
  by the highest stock price (e.g. the \texttt{High} variable in the Apple data),
  Use R to estimate a simple linear regression model for the Apple data where the response
  variable (this is the ``y-variable'' that you are trying to explain or model) is
  \texttt{appleClose} and the explanatory variable (this is the ``x-variable'') is \texttt{appleHigh}, that is,
  estimate the regression line:

  \begin{equation*}
    \mathtt{appleClose} = \hat \alpha + \hat \beta \cdot \mathtt{appleHigh} 
  \end{equation*}
  and assign your regression model object to \texttt{appleModel}.

\item Inspect the \texttt{appleModel} object and answer the following questions.
  \begin{enumerate}
  \item What are the estimated model coefficients?
    
  \item What are the standard errors for the regression coefficients?
 
  \item Make a histogram of the model residuals.

  \item Plot the model object.

  \end{enumerate}

\item The open price of the stock market presumably will also affect the
  close price. Extract the open price from the apple data and name it as
  \texttt{appleOpen}. Estimate the multiple regression model 
 
  \begin{equation*}
    \mathtt{appleClose} = \hat \alpha + \hat \beta \cdot \mathtt{appleHigh} +
    \hat \gamma \cdot \mathtt{appleOpen} 
  \end{equation*}
  and save your regression model object as \texttt{appleModel2}.

 \item Study \texttt{appleModel2} and answer the questions in \textbf{5.2}. Do
   you get a better model?

 \item The dataset \texttt{Apple} contains the stock prices for 1894 days. Change the
   arguments of \texttt{lm()} so that the model in \textbf{5.3} is re-estimated using only the most
   recent $100$ days of data (that is, the first $100$ rows of the dataset). 

 \item Actually, all three variables \texttt{Open}, \texttt{Low}, and
   \texttt{High} can possibly be explanatory variables in the model for \texttt{Close}. But there are a
   lots of combinations to choose from (only \texttt{Open}, only \texttt{Low}, both of
   \texttt{Open} and \texttt{Low} etc etc). Use forward stepwise regression to select the
   best set of explanatory variables. Use the full data set.

 \item Do you end up with the same model if you use backward stepwise regression in
   \textbf{5.6}?

\end{enumerate}


\end{document}
