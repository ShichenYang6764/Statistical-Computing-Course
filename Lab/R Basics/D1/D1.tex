\documentclass[a4paper]{article}
\usepackage[a4paper]{geometry}
\geometry{verbose,tmargin=2.5cm,bmargin=2.5cm,lmargin=2.5cm,rmargin=2.5cm}

\usepackage{fancyhdr}
%The first page setting
\fancypagestyle{plain}
{%
  \fancyhf{} % clear all header and footer fields
  \fancyhead[L]{
    Central University of Finance and Economics\\
    School of Statistics and Mathematics\\
    Feng Li
  }
  \fancyhead[R]{Programming in R}
}
%The remaining pages
\pagestyle{fancy}
\fancyhead[RO,LE]{}
\fancyhead[C]{Programming in R}
\fancyhead[LO,RE]{}


\title{D1--Introduction to Computers, Programming and R}
\date{\small{Revised on \today}}

\begin{document}
\maketitle
\hrule

\section{}
\begin{enumerate}
\item Start \textbf{R}.
\item You are now at the prompt of \textbf{R}. Read the welcome information and
  find out how to quit \textbf{R} properly.
\item Start \textbf{R} again and treat it as a calculator to do the following
  calculations
\begin{verbatim}
3 + 4
5 * 6
12/7
2^3
45-2*3
(45-2)*3
\end{verbatim}
\item Do the above in another way like this
\begin{verbatim}
a <- 3
b <- 4
c <- a + b
\end{verbatim}
Now check what \texttt{a,b,c} are. You can just type e.g. \texttt{a} and RETURN
or use the command \texttt{print(a)}.

\item Type \texttt{quit()} and answer \texttt{no}.
\item Start \textbf{R} and check if \texttt{a,b,c} are still there.
\item Redo \textbf{6.} and then quit \textbf{R} by answering \texttt{yes}.
\item Start \textbf{R} and check if \texttt{a,b,c} are still there. What did
  you see?
\item What do \texttt{yes} and \texttt{no} mean when we use the \texttt{quit()}
  command? You may check the help of \texttt{quit()}. Type \texttt{?quit} at
  the \textbf{R} prompt and use the \textbf{PgUp} and \textbf{PgDn} keys on the keyboard to
  navigate. To quit the current \textbf{R} help in the terminal, type \texttt{q}.
\item $\heartsuit$ Discussion: for a computer to complete a task like procedure
  \textbf{1.9}--\textbf{1.10} what major role do CPU, memory, hard drive,
  keyboard and screen play?
\end{enumerate}
\section{}
\begin{enumerate}
\item Check what variables you have in the current \textbf{R} session, type \texttt{ls()}.
\item Remove all of those variables. [Hint: type \texttt{??remove} to search
  possible matches. Read the help for individual functions before you issue any
  command.]
\item $\heartsuit$ What's the difference between ``\texttt{?}'' and ``\texttt{??}''? [Hint: try
  \texttt{help("?")} and \texttt{help("??")}]
\item What other ways of help are available in \textbf{R}? How to launch the html version of \textbf{R} help documentation? [Hint: read \textbf{2.3}
  carefully.]
\end{enumerate}

\section{}
\begin{enumerate}
\item $\heartsuit$ How many ways can you find to assign \texttt{5} to variable
  \texttt{myVar}? Show with examples.
\item $\heartsuit$ What's the difference between ``\texttt{<-}'' and ``\texttt{=}"? Show
  with examples.
\item  Try the following commands line by line and discuss their implications. [Hint: Be careful of the space you are typing.]
\begin{verbatim}
a <- 5
print(a)
a < - 7
print(a)
rm(a)
a < - 7
\end{verbatim}
\item Read the help of the function \texttt{print()} and try
\begin{verbatim}
x <- "The value of pi is"
print(x)
print(pi)
\end{verbatim}
\item Repeat the above commands but replace \texttt{print} with
  \texttt{cat}. Use the \texttt{cat()} function, \texttt{x} and \texttt{pi} to print the
  following sentence to the screen:
\begin{verbatim}
The value of pi is: 3.141593.
\end{verbatim}
\item Repeat the above task but with the function \texttt{message()}. What are
  the differences between \texttt{print}, \texttt{cat}, and \texttt{message} ?
\end{enumerate}

\section{}
\begin{enumerate}
\item Create the following vectors:
\begin{verbatim}
myVar1 <- c(1,2,3,4,5)
myVar2 <- c("Mon","Tue","Wed","Thu","Fri", "Sat", "Sun")
myVar3 <- c(TRUE, TRUE, FALSE, FALSE, FALSE)
myVar4 <- c(0.2,0.4,0.6,0.8,1.0)
myVar5 <- c("Jim","Apple","Linda","School","Math")
myVar6 <- c(2,3,6,7,8,9)
\end{verbatim}
\item  $\heartsuit$ How many elements do you have in each vector? [Hint \texttt{?length}]
\item What types of vectors are they? [Hint \texttt{?mode}]
\item Do some basic calculations e.g.
\begin{verbatim}
myVar1 + myVar4
myVar1 - myVar4
myVar1 * myVar4
myVar1 / myVar4
(myVar1 - myVar4)^2
myVar1^2 - exp(myVar4)
log(myVar1)
log(myVar1) + 1
\end{verbatim}
and speculate on why obtained those results.
% \item Repeat the above calculations but replace \texttt{myVar1} with
%   \texttt{myVar5}. What did you see and what was the reason for that? What if you replace \texttt{myVar1} with
%   \texttt{myVar6} or \texttt{myVar3}?
% \item In the above example, you will probably get warnings or error
%   messages. Are all of those harmless/harmful?
% \item Try the following commands
% \begin{verbatim}
% myVarNew1 <- c(myVar1, myVar4, myVar6)
% myVarNew2 <- c(myVar1, myVar2)
% myVarNew3 <- c(myVar2, myVar5)
% myVarNew4 <- c(myVar1, myVar3)
% myVarNew5 <- c(myVar2, myVar3)
% \end{verbatim}
% and check the type of new variables. Are they still the same as the old ones?
\end{enumerate}

\section{}
\begin{enumerate}
\item Assume you have the variable \texttt{myVar6} and you want to replace the
  fourth element \texttt{7} with \texttt{7000}. How can you manage that? [Hint: Use
  the \texttt{fix()} or \texttt{edit()} functions].
\item Print your new \texttt{myVar6} to check if the variable has been updated.
\item Use the \texttt{getwd()} to check the location of your current directory, and
  assign the current working directory path to a variable \texttt{myOldDir}.
\item Use \texttt{save.image()} to save all your available variables to a file
  named ``\texttt{myOldFullVars.RData}'' under current directory.
\item Use \texttt{dir()} function to check if you have saved the file
  successfully.
\item Switch your working directory to a temporary directory ``/tmp/'' via the
  function \texttt{setwd()}.
\item Use \texttt{getwd()} again to check if your working directory have been
  switched successfully.
\item Again use \texttt{ls()} to list all the available variables.
\item Use \texttt{save()} to only save some variables to a file
  ``\texttt{myNewVars.RData}'' under current directory.
\item Quit \textbf{R} and start \textbf{R} again.
\item Clear all the variables with the command \texttt{rm(list=ls())}.
\item Load the file ``\texttt{myOldFullVars.RData}'' using the function
  \texttt{load()} and use \texttt{ls()} and \texttt{print()} to check if your
  old variables are all right.
\item  $\heartsuit$ Clear all the variables again and load the file
  ``\texttt{myNewVars.RData}''. Do you still have the variables that you saved? If not, why?
\end{enumerate}

% \section{}
% \begin{enumerate}
%  \item Quit \textbf{R} and type \texttt{jgr} on your terminal and hint RETURN. You will be
%   guided into an integrate development environment (IDE) for R.
% \item Redo problem \textbf{5.1-5.13} interactively to familiarize yourself with the environment.
% \end{enumerate}

\section{}
\begin{enumerate}
%\item Launch \texttt{jgr} and clear the \textbf{R} session.
\item $\heartsuit$ Open the R editor (File $\to$ New) and save it as
  \texttt{D1-FirstName.LastName.R} Write your R code to the following questions (\textbf{7.3}-\textbf{7.8}).
\item Use \texttt{read.table()} to read dataset ``\texttt{Apple.txt}'' and name it as
  ``\texttt{Apple}''. The data file is included in the lab assignment. Place
  the file in your current working directory so that \textbf{R} can find it. [Hint: Use
 the arguments \texttt{header} and \texttt{sep}. Look it up in the help function for read.table]
\item Print the variable and check the data type.
\item Export/Save the variable ``\texttt{Apple}'' into the \textbf{R} data format as ``\texttt{Apple.RData}''.
\item Repeat the above 1.--4. for ``\texttt{NASDAQ100.csv}''. [Hint: try \texttt{read.csv()}].
\item Use an external program (e.g. office suit) to open the file
  ``\texttt{Google.xls}'' and save it as \texttt{Google.csv}. Repeat the loading and
  exporting procedure.
\item Load the \textbf{R} datasets you have converted and export them to \texttt{txt}
  and \texttt{csv} formats. [Hint: try \texttt{dput()} and \texttt{write.csv()}
  functions]. Attach the dataset file ``\texttt{NASDAQ100.txt}'' in your submitted report.
\end{enumerate}

\section{Extra}
Installing \textbf{R} on your own computer is simple and free.
\begin{enumerate}
\item If you use Microsoft Windows, visit
  \texttt{http://www.r-project.org/}. It shipped with a
  simple graphical user interface (GUI).
\item R is available for Mac OS X users with a nice build-in graphical user interface, visit \\
\texttt{http://www.r-project.org}
\item If you use Linux, search the phrase ``r-cran'' in your system's software
  repository. Any text editor can be used to edit R source code. But if you
  want a graphical user interface, RStudio (with is easy to install and also available for other
  platforms) is a good start. Please visit \texttt{http://www.rstudio.org/download/desktop}.

\end{enumerate}
\end{document}
